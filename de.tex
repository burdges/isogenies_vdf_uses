\documentclass{llncs}

\usepackage{amsmath,amsfonts,amssymb}
\usepackage{xcolor}
\usepackage{hyperref}
\hypersetup{
	unicode=true,
	colorlinks=true,
	citecolor=blue!70!black,
	filecolor=black,
	linkcolor=red!70!black,
	urlcolor=blue,
	pdfstartview={FitH},
}

\newcommand{\Z}{\mathbb{Z}}
\newcommand{\F}{\mathbb{F}}
\renewcommand{\O}{\mathcal{O}}
\DeclareMathOperator{\End}{End}
\DeclareMathOperator{\poly}{poly}
\DeclareMathOperator{\polylog}{polylog}
\DeclareMathOperator{\Setup}{\mathsf{Setup}}
\DeclareMathOperator{\TSetup}{\mathsf{TrustedSetup}}
\DeclareMathOperator{\Extract}{\mathsf{Extract}}
\DeclareMathOperator{\Encaps}{\mathsf{Encaps}}
\DeclareMathOperator{\Decaps}{\mathsf{Decaps}}
%\newcommand{\pp}{\mathsf{pp}}
\newcommand{\ek}{\mathsf{ek}}
\newcommand{\pk}{\mathsf{pk}}
\newcommand{\id}{\mathsf{id}}
\newcommand{\idk}{\mathsf{idk}}
\newcommand{\keyspace}{\mathcal{K}}
\newcommand{\cipherspace}{\mathcal{C}}
\newcommand{\Emid}{E_\mathrm{mid}}
\newcommand{\Qmid}{Q_\mathrm{mid}}

\title{Delay Encryption}
\author{Jeffrey Burdges\inst{1}
  \and
  Luca De Feo\inst{2}\orcidID{0000-0002-9321-0773}}
\institute{
  Web 3, Switzerland
  \and
  IBM Research Zürich, Switzerland
}

\begin{document}

\maketitle

\begin{abstract}
  We introduce a new primitive named Delay Encryption, and give an
  efficient instantation based on isogenies of supersingular curves
  and pairings. %
  Delay Encryption is related to Time-lock Puzzles and Verifiable
  Delay Functions, and can be roughly described as ``time-lock
  identity based encryption''. %
  It has several applications in distributed protocols, such as
  sealed bid Vickrey auctions and electronic voting.

  We give an instantiation of Delay Encryption by modifying Boneh and
  Frankiln's IBE scheme, where we replace the master secret key by a
  long chain of isogenies, as in the isogeny VDF of De Feo, Masson, 
  Petit and Sanso. %
  Similarly to the isogeny-based VDF, our Delay Encryption requires a
  trusted setup before parameters can be safely used; our trusted
  setup is identical to that of the VDF, thus the same parameters can
  be generated once and shared for many executions of both protocols,
  with possibly different delay parameters.

  We also discuss several topics around delay protocols
  based on isogenies that were left untreated by De Feo \emph{et al.},
  namely: distributed trusted setup, watermarking, and implementation
  issues.
\end{abstract}

\section{Introduction}
\label{sec:introduction}

The first appearance of \emph{delay cryptography} was in Rivest,
Shamir and Wagner's~\cite{TLP} \emph{Time-lock Puzzle}, an encryption
primitive where the holder of a trapdoor can encrypt (or decrypt)
``fast'', but where anyone not knowing the trapdoor can only decrypt
(or encrypt) ``slowly''.

Recently, a revival of delay cryptography has been promoted by
research on blockchains, in particular thanks to the introduction of
\emph{Verifiable Delay Functions (VDF)}~\cite{Boneh}: deterministic
functions $f$ that can only be evaluated  ``sequentially'' and ``slowly'',
but such that verifying that $y=f(x)$ is ``fast''. %

After their definition, VDFs quickly gained attention, prompting two
independent solutions in the space of a few
weeks~\cite{Wesolowski,Pietrzak}. %
Both proposals are based on repeated squaring in groups of unknown
order, and are similar in spirit to Rivest \emph{et al.}'s time-lock
puzzle, however they use no trapdoor.

One year later, another VDF, based on a different algebraic structure,
was proposed by De Feo, Masson, Petit and
Sanso~\cite{10.1007/978-3-030-34578-5_10}. %
This VDF uses chains of supersingular isogenies as ``sequential slow''
functions, and pairings for efficient verification. %
Interestingly, it is not known how to build a time-lock puzzle from
isogenies; in this work we introduce a new primitive, in some respects
more powerful than time-lock puzzles, that we are able to instantiate
from isogenies.

\subsubsection{Limitations of Time-lock puzzles.}
Time-lock puzzles allow one to ``encrypt to the future'', i.e., to
create a puzzle $\pi$ that encapsulates a message $m$ for a set amount
of time $T$. %
They have the following two properties:
\begin{itemize}
\item Puzzle generation is efficient: there exists an algorithm which,
  on input the message $m$ and the \emph{delay} $T$, generates $\pi$
  in time much less than $T$.
\item Puzzle solving is predictably slow and sequential: on input
  $\pi$, the message $m$ can be recovered by a circuit of depth
  approximately $T$, and no circuit of depth less than $T$ can
  recover $m$ reliably.
\end{itemize}

Time-lock puzzles can be used to remove trusted parties from
protocols, replacing them with a time-consuming puzzle solving. %
Prototypical applications are auctions and electronic voting, we will
use auctions as a motivating example.

In a highest bidder auction, the easy solution in presence of a
trusted authority is to encrypt bids to the authority, who then
decrypts all the bids and selects the winner. %
Lacking a trusted authority, the standard solution is to divide the
auction in two phases: in the \emph{bidding phase} all bidders commit
to their bids using a commitment; in the \emph{tallying phase} bidders
open their commitments, and the highest bidder wins. %
However, this design has one flaw in contexts where it is required
that all bidders reveal their bids at the end of the auction. %
For example, in \emph{Vickrey auctions}, the highest bidder wins the
auction, but only pays the price of the second highest bid. %
If at the end of the auction some bidders refuse to open their
commitment, the result of the auction may be invalid.

Time-lock puzzles solve this problem: by having bidders encapsulate
their bid in a time-lock puzzle, it is guaranteed that all bids can be
decrypted in the tallying phase. %
However this solution becomes very expensive in large auctions,
because one puzzle per bidder must be solved: if several thousands of
bidders participate, the tallyers must strike a balance between
running thousands of puzzle solving computations in parallel, and
having a tallying phase that is thousands of times longer than the
bidding phase. %
Since time-lock puzzles use trapdoors for puzzle generation, a
potential mitigation is to have the bidders reveal their trapdoors in
the tallying phase, thus speeding up decryption; however this does not
help in presence of a large number of uncollaborative bidders.

An elegant solution introduced in~\cite{C:MalThy19} is to use
\emph{Homomorphic Time-lock Puzzles (HTLP)}, i.e., time-lock puzzles
where the puzzles can be efficiently combined homomorphically. %
Using these, the tallyers can efficiently evaluate the desired
tallying circuit on the unopen puzzles, and then run only a single
slow puzzle-solving algorithm. %
Unfortunately, the only efficient HTLPs introduced
in~\cite{C:MalThy19} are simply homomorphic (either additively or
multiplicatively), and they are thus only useful for voting; fully
homomorphic TLPs, which are necessary for auctions, are only known
from indistinguishability obfuscation~\cite{FOCS:GGHRSW13}, and are
thus unpractical. %

On top of that, it can be argued that time-lock puzzles are not the
appropriate primitive to solve the problem: why do the tallyers need
to run one of two different algorithms to open the puzzles? Are
trapdoors really necessary? %
In this work, we introduce a new primitive, \emph{Delay Encryption},
that arguably solves the problem more straightforwardly and elegantly.


\subsubsection{Delay Encryption.}
Delay Encryption is related to both Time-lock puzzles and VDFs,
however it does not seem to be subsumed by either. %
It can be viewed as a time-lock analogue of Identity Based Encryption,
where the derivation of individual private keys is sequential and
slow. %

Instead of senders and receivers, Delay Encryption has a concept of
\emph{sessions}. %
A session is defined by a \emph{session identifier}, which must be a
hard to predict string. %
When a session identifier $\id$ is issued, anyone knowing $\id$ can
\emph{encrypt to the session for $\id$}; decryption is however
unfeasible without a \emph{session key}, which is to be derived from
$\id$. %
The defining feature of Delay Encryption is \emph{extraction}: the
process of deriving a session key from a session identifier. %
Extraction must be a sequential and slow operation, designed to run in
time $T$ and no less for almost any $\id$.

Since there are no secrets in Delay Encryption, anyone can run
extraction. %
It is thus important that session identifiers are hard to predict, and
thrown away after the first use, otherwise an attacker may precompute
session keys and immediately decrypt any ciphertext to the sessions.

Delay Encryption is different from known Time-lock puzzles in that it
has no trapdoor, and from VDFs in that it provides a fast encryption,
rather than just a fast verification. %
It has similar applications to Homomorphic time-lock puzzles, it is
however more efficient and solves many problems more
straightforwardly.


\subsubsection{Applications of Delay Encryption.}

We already mentioned the two main applications of Time-lock puzzles. %
We review here how Delay Encryption offers better solutions.

\paragraph{Vickrey auctions.} 
Sealed bid auctions are easily implemented with standard commitments:
in the bidding phase each bidder commits to its bid; later, in the
tallying phase each bidder opens their commitment. %
However this solution is problematic when some bidders may refuse to
open their commitments.
   
Delay Encryption provides a very natural solution: at the beginning of
the auction an \emph{auction id} is selected using some unpredictable
and unbiased randomness, e.g., coming from a randomness beacon. %
After the auction id is published, all bidders encrypt to the auction
as senders of a Delay Encryption scheme. %
In the meantime, anyone can start computing the auction key using the
extraction functionality. %
When the \emph{auction key} associated with the auction id is known,
anyone in possession of it can decrypt all bids and determine the winner.

\paragraph{Electronic voting.} In electronic voting it is often
required that the partial tally of an election stays unknown until the
end, to avoid influencing the outcome. %

Delay Encryption again solves the problem elegantly: once the
\emph{election id} is published, all voters can cast their ballot by
encrypting to it. %
Only after the election key is published, anyone can verify the
outcome by decrypting the ballots. %

Of course this idea can be combined with classical techniques for
anonymity, integrity, etc.

\medskip

In both applications it is evident that the session/auction/election
id must be unpredictable and unbiased: if it is not, someone may start
computing the session key before anyone else can, and thus break the
delay property. %
Fortunately, this requirement is easily satisfied by using randomness
beacons, which, conveniently, can be implemented using VDFs.

\subsubsection{Plan.}
We start by defining Delay Encryption in
Section~\ref{sec:definitions}, and give our instantiation in
Section~\ref{sec:delay-encrypt-from}. %
In the following sections, we discuss several topics related to both
Delay Encryption and VDFs from isogenies and pairings:
Section~\ref{sec:distr-trust-setup} explains how to efficiently
implement the trusted setup, common to both our Delay encryption and
the isogeny based VDF, in a distributed manner;
Section~\ref{sec:watermarking} covers Watermarking, a mechanism to
prove ``ownership'' of a ``slow'' computation; finally
Section~\ref{sec:secure-impl-isog} discusses implementation details
and challenges for isogeny based delay functions.


\section{Definitions}
\label{sec:definitions}

Our definition of Delay Encryption uses an API similar to a Key
Encapsulation Mechanism; the adaptation to a PKE-like API is
straightforward. A Delay Encryption scheme consists of four
algorithms: $\Setup$, $\Extract$, $\Encaps$ and $\Decaps$:

\begin{description}
\item[$\Setup(\lambda, T) \to (\ek,\pk)$.] %
  Takes a \emph{security parameter} $\lambda$, a \emph{delay
    parameter} $T$, and produces public parameters consisting of an
  \emph{extraction key} $\ek$ and an \emph{encryption key} $\pk$. %
  $\Setup$ must run in time $\poly(\lambda,T)$.
\item[$\Extract(\ek,\id) \to \idk$.] %
  Takes the extraction key $\ek$ and a \emph{session identifier}
  $\id\in\{0,1\}^*$, and outputs a \emph{session key} $\idk$. %
  $\Extract$ is expected to run in time \emph{exactly} $T$, see below.
\item[$\Encaps(\pk,\id)\to (c,k)$.] %
  Takes the encryption key $\pk$ and a \emph{session identifier}
  $\id\in\{0,1\}^*$, and outputs a \emph{ciphertext}
  $c\in\cipherspace$ and a \emph{key} $k\in\keyspace$. %
  $\Encaps$ must run in time $\poly(\lambda)$.
\item[$\Decaps(\pk,\id,\idk,c)\to k$.] %
  Takes the encryption key $\pk$, a \emph{session identifier}
  $\id$, a \emph{session key} $\idk$, a ciphertext $c\in\cipherspace$,
  and outputs a key $k\in\keyspace$. %
  $\Decaps$ must run in time $\poly(\lambda)$.
\end{description}

A Delay Encryption scheme is correct if for any
$(\ek,\pk)=\Setup(\lambda,T)$ and any $\id$
\[\idk=\Extract(\ek,\id)
  \;\wedge\;
  (c,k) = \Encaps(\pk,\id)
  \;\Rightarrow\;
  \Decaps(\pk,\id,\idk,c) = k.\]
The security of Delay Encryption is defined similarly to that of
public key encryption schemes, and in particular of identity-based
ones; however one additional property is required of $\Extract$: that
for a randomly selected identifier $\id$, the probability that any
algorithm outputs $\idk$ in time less than $T$ is negligible. %
We now give the formal definition.

\paragraph{The security game.} It is apparent from the definitions
that Delay Encryption has no secrets: after public parameters $(\ek,\pk)$
are generated, anyone can run any of the algorithms. %
Thus, the usual notion of indistinguishability will only be defined
with respect to the delay parameter $T$: no adversary is able to
distinguish a key $k$ from a random string in time $T-o(T)$, but
anyone can in time $T$. %
Properly defining what is meant by ``time'' requires fixing a
computation model. %
Here we follow the usual convention from VDFs, and assume a model of
parallel computation: in this context, ``time $T$'' may mean $T$ steps
of a parallel Turing machine, or an arithmetic circuit of depth $T$. %
Crucially, we do not bound the amount of parallelism of the Turing
machine, or the breadth of the circuit, i.e., we focus on
\emph{sequential delay} functions.

We consider the following $\Delta$-IND-CPA game. %
Note that the game involves no oracles, owing to the fact that the
scheme has no secrets. %
%
\begin{description}
\item[Precomputation.] The adversary receives $(\ek,\pk)$ as input, and
  outputs an algorithm $\mathcal{D}$. %
\item[Challenge.] The challenger selects a random $\id$ and computes
  a key encapsulation $(c,k_0)\gets\Encaps(\pk,\id)$. %
  It then picks a uniformly random $k_1\in\keyspace$, and a random bit
  $b\in\{0,1\}$. %
  Finally, it outputs $(c,k_b,\id)$.
\item[Guess.]  Algorithm $\mathcal{D}$ is run on input
  $(c,k_b,\id)$. %
  The adversary wins if $\mathcal{D}$ terminates in time less than
  $\Delta$, and the output is such that $\mathcal{D}(c,k_b,\id) = b$.
\end{description}

We stress that the game is intrinsically non-adaptive, in the sense
that no computation is ``free'' after the adversary has seen the
challenge.

We say a Delay Encryption scheme is \emph{$\Delta$-Delay
  Indistinguishable under Chosen Plaintext Attacks} if any
adversary running the precomputation in time
$\poly(\lambda,T)$ has negligible advantage in winning the game. %
Obviously, the interesting schemes are those where $\Delta = T-o(T)$.

\begin{remark}
  Although it would be possible to define an analogue of chosen
  ciphertext security for Delay Encryption, by giving algorithm
  $\mathcal{D}$ access to a decryption oracle for $\id$, it is not
  clear what kind of real world attacks this security notion could
  model. Indeed, an instantaneous decryption oracle for $\id$ would go
  against the idea that the session key $\idk$ needed for decryption
  is not known to anyone before time $T$.

  Similarly, one could imagine giving $\mathcal{D}$ access to an
  extraction oracle, to allow it instantaneous adaptive extraction
  queries after the challenge (note that in the precomputation phase
  the adversary is free to run polynomially many non-adaptive
  extractions). However it is not clear what component of a real world
  system could provide such instantaneous extraction in practice,
  since extraction is a public (and slow) operation.
\end{remark}


\subsection{Relationship with other primitives}

\paragraph{Delay Encryption and Identity Based Encryption.}
Although there is no formal relationship between Identity Based
Encryption (IBE) and Delay Encryption, the similarity is evident. 

Recall that an IBE scheme is a public key encryption with three
parties: a dealer who possesses a master private/public key pair, a
receiver who has an \emph{identity} that acts as its public key (e.g.,
its email address), and a sender who wants to send a message to the
receiver. %
In IBE, the dealer runs an \emph{extraction} on the identity to
generate the receiver's secret key. %
The sender encrypts messages to the receiver using both the identity
and the master public key. %
The receiver decrypts using the master public key and the private key
provided by the dealer.

Delay Encryption follows the same blueprint, but has no secrets:
there is no master key anymore, but only a set of public parameters $(\ek,\pk)$. %
Receiver identities become session identifiers $\id$: public but
unpredictable. %
The dealer is replaced by the public functionality
$\Extract(\ek,\id)$: sequential and slow. %
Senders encrypt messages to the sessions by using $\pk$ and $\id$. %
After extraction has produced $\idk$, anyone can decrypt messages
``sent'' to $\id$ by using $\idk$.

The similarity with IBE is not fortuitous. %
Indeed, the instantiation we present next is obtained from Boneh and
Franklin's IBE scheme~\cite{doi:10.1137/S0097539701398521}, by
replacing the master secret with a public, slow to evaluate, isogeny.
This is analogous to the way De~Feo \emph{et al.}'s
VDF~\cite{10.1007/978-3-030-34578-5_10} is obtained from the
Boneh--Lynn--Shacham signature scheme~\cite{boneh+lynn+shacham04}.

The similarity with IBE will be mirrored both in the reductions we
discuss next, and in the security proof of our instantiation.

\paragraph{Delay Encryption and Verifiable Delay Functions.}
Boneh and Franklin attribute to Naor the observation that IBE implies
signatures. %
The construction is straightforward: messages are encoded to
identities; to sign a message $\id$, simply output the derived private
key $\idk$ associated to it. %
To verify a signature $(\id,\idk)$: run encapsulation to $\id$
obtaining a random $(c,k)$, decapsulate $(\id,\idk,c)$ to obtain $k'$,
and accept the signature if $k=k'$. %
The signature scheme is existentially unforgeable if the IBE scheme is
indistinguishable under chosen ciphertext attacks.

Precisely the same construction shows that Delay Encryption implies
(sequential) Proof of Work. %
Furthermore, if we define \emph{extraction soundness} as the property
that adversaries have negligible chance of finding $\idk\ne\idk'$ such
that
\[\Decaps(\pk,\id,\idk,c) = \Decaps(\pk,\id,\idk',c),\]
then we see that extraction sound Delay Encryption implies Verifiable
Delay Functions. %
It is easily verified that the derived VDF is $\Delta$-sequential if
the Delay Encryption scheme is $\Delta$-IND-CPA.

The signature scheme derived from Boneh and Franklin's IBE is
equivalent to the Boneh--Lynn--Shacham scheme. %
Unsurprisingly, the instantiation of Delay Encryption that we give in
the next section is extraction sound, and the derived VDF is
equivalent to De~Feo \emph{et al.}'s VDF.

\paragraph{Delay Encryption and Time-lock puzzles.}
Both Delay Encryption and Time-lock puzzles permit a form of
\emph{encryption to the future}: encrypt a message now, so that it can
only be decrypted at a set time in the future. %
There is no formal definition of Time-lock puzzles commonly agreed
upon in the literature, it is thus difficult to say what they exactly
are and how they compare to Delay Encryption.

Bitansky \emph{et al.}~\cite{10.1145/2840728.2840745} define 
Time-lock puzzles as two algorithms
\begin{itemize}
\item $\mathsf{Gen}(\lambda,T, s) \to Z$ that takes as input a delay parameter
  $T$ and a solution $s\in\{0,1\}^\lambda$, and outputs a puzzle $Z$;
\item $\mathsf{Solve}(Z) \to s$ that takes as input a puzzle $Z$ and
  outputs the solution $s$;
\end{itemize}
under the constraints that $\mathsf{Gen}$ runs in time
$\poly(\lambda,\log T)$ and that no algorithm computes $s$ from $Z$ in
parallel time significantly less than $T$.

One might be tempted to construct a Time-lock puzzle from Delay
Encryption by defining $\mathsf{Gen}$ as follows:
\begin{enumerate}
\item Compute $(\ek,\pk) \gets \Setup(\lambda,T)$;
\item Sample a random $\id\in\{0,1\}^\lambda$;
\item Compute $(c,k) \gets \Encaps(\pk, \id)$;
\item Compute $m = E_k(s)$;
\item Return $(\ek,\pk,\id,c,m)$;
\end{enumerate}
where $E_k$ is a symmetric encryption scheme. %
Then $\mathsf{Solve}$ is naturally defined as
\begin{enumerate}
\item Compute $\idk \gets \Extract(\ek,\id)$;
\item Compute $k \gets \Decaps(\pk,\id,\idk,c)$;
\item Return $s = D_k(m)$;
\end{enumerate}
where $D_k$ is the decryption routine associated to $E_k$. %

However this fails to define a Time-lock puzzle in the sense above,
because $\Setup$ can take time $\poly(\lambda,T)$ instead of
$\poly(\lambda,\log T)$. %
If we take $\Setup$ out of $\mathsf{Gen}$, though, we obtain something
very similar to what Bitansky \emph{et al.} call Time-lock puzzles
\emph{with pre-processing}, albeit slightly weaker.%
\footnote{Bitansky \emph{et al.} require pre-processing to run in
  sequential time $T\cdot\poly(\lambda)$, but parallel time only
  $\poly(\lambda,\log T)$.}.

We see no technical obstruction to having $\Setup$ run in time
$\poly(\lambda,\log T)$, and thus being a strictly stronger primitive
than Time-lock puzzles. %
However our instantiation does not satisfy this stronger notion of
Delay Encryption, and, lacking any other candidate, we prefer to keep
our definitions steeping in reality.

\medskip

To summarize, Delay Encryption is a natural analogue of Identity Based
Encryption in the world of time delay cryptography. %
It requires Proofs of Work to exist, and a mild strengthening of it
(which we are able to instantiate) implies Verifiable Delay
Functions. %
It also implies a weak form of Time-lock puzzles, and a strengthening
of it (of which we know no instantiation) implies standard Time-lock
puzzles. %
At the same time, no dependency is known between Time-lock puzzles and
Verifiable Delay Functions, indicating that Delay Encryption is
possibly a stronger primitive than both.


\section{Delay Encryption from isogenies and pairings}
\label{sec:delay-encrypt-from}

We instantiate Delay Encryption from the same framework De Feo,
Masson, Petit and Sanso used to instantiate Verifiable Delay
Functions~\cite{10.1007/978-3-030-34578-5_10}. %
We briefly recall it here for completeness.

An elliptic curve $E$ over a finite field $\F_{p^n}$ is said to be
supersingular if the trace of its Frobenius endomorphism is divisible
by $p$, i.e., if $\#E(\F_{p^n})=1\mod p$. %
Over the algebraic closure of $\F_p$, there is only a finite number of
isomorphism classes of supersingular curves, and every class contains
a curve defined over $\F_{p^2}$. %

We will only work with supersingular curves $E/\F_{p^2}$ whose group
of $\F_{p^2}$-rational points is isomorphic to ${(\Z/(p+1)\Z)}^2$. %
For these curves, if $N$ is a divisor of $p+1$, we will denote by
$E[N]$ the subgroup of $\F_{p^2}$-rational points of order $N$, which
is isomorphic to ${(\Z/N\Z)}^2$.

However, among these curves some will be curves $E/\F_p$ defined over
$\F_p$, seen as curves over $\F_{p^2}$ (in algebraic jargon, with
scalars extended from $\F_p$ to $\F_{p^2}$). %
For this special case, if $N$ is an odd divisor of $p+1$, the
$\F_{p^2}$-rational torsion subgroup $E[N]$ contains two distinguished
subgroups: the subgroup $E[N]\cap E(\F_p)$ of points of order $N$
defined over $\F_p$, which we also denote by $E[(N,\pi-1)]$; and the
subgroup of points of order $N$ not in $E(\F_p)$, but with
$x$-coordinate in $\F_p$, which we denote by $E[(N,\pi+1)]$.

An isogeny is a group morphism of elliptic curves with finite
kernel. %
In particular, isogenies preserve the group order of elliptic curves,
and thus they preserve supersingularity. %
Isogenies can be represented by ratios of polynomials, and, like
polynomials, have a \emph{degree}. %
Isogenies of degree $\ell$ are also called $\ell$-isogenies; the
degree is multiplicative with respect to composition, thus
$\deg\phi\circ\psi=\deg\phi\cdot\deg\psi$. %
The degree is an important invariant of isogenies, roughly measuring
the amount of information needed to represent them.

An isogeny graph is a graph whose vertices are isomorphism classes of
elliptic curves, and whose edges are isogenies, under some
restrictions. %
Isogeny-based cryptography mainly uses two types of isogeny graphs:
\begin{itemize}
\item The \emph{full supersingular graph} of $\F_p$, whose vertices
  are all isomorphism classes of supersingular curves over $\F_{p^2}$,
  and whose edges are all isogenies of a prime degree $\ell$;
  typically $\ell=2,3$.
\item The \emph{$\F_p$-restricted supersingular graph}, or
  \emph{supersingular CM graph} of $\F_p$, whose vertices are all
  $\F_p$-isomorphism classes of supersingular curves over $\F_p$, and
  whose edges are $\ell$-isogenies for all primes $\ell$ up to some
  bound; typically $\ell\lessapprox\lambda\log\lambda$, where
  $\lambda$ is the security parameter.
\end{itemize}

Any $\ell$-isogeny $\phi:E\to E'$ has a unique \emph{dual}
$\ell$-isogeny $\hat\phi:E'\to E$ such that
\begin{equation}
  \label{eq:adjoin}
  e_N'(\phi(P),Q) = e_N(P,\hat\phi(Q)),
\end{equation}
for any integer $N$ and any points $P\in E[N]$, $Q\in E'[N]$, where
$e_N$ is the Weil pairing on $E$, and $e'_N$ the one on $E'$. %
The same equation, with the same $\hat\phi$, also holds for any other
known pairing, such as the Tate and Ate pairings.

The framework of De Feo \emph{et al.} uses chains of small degree
isogenies as delay functions, and the pairing
equation~\eqref{eq:adjoin} as an efficient means to verify the
computation. %
Formally, they propose two related instantiations of VDF, following
the same pattern: they both use the same base field $\F_p$, where $p$
is a prime of the form $p+1=N\cdot f$, chosen so that discrete
logarithms in the group of $N$-th roots of unity in $\F_{p^2}$ (the
target group of the pairing) are hard (i.e., $N\approx 2^{2\lambda}$
and $p\sim 2^{\lambda^3}$). %
They have a common trusted setup, independent of the delay parameter,
and the usual functionalities of a VDF:
\begin{description}
\item[Trusted setup] selects a random supersingular elliptic curve $E$
  over $\F_p$.
\item[Setup] takes as input $p,N,E$, a delay parameter $T$, and
  performs a walk in an $\ell$-isogeny graph to produce a degree
  $\ell^T$ isogeny $\phi:E\to E'$.
  
  It also computes a point $P\in E(\F_p)$ of order $N$. %
  It  outputs $\phi,E',P,\phi(P)$.
\item[Evaluation] takes as input a random point $Q\in E'[N]$ and outputs
  $\hat\phi(Q)$.
\item[Verification] uses Eq.~\eqref{eq:adjoin} to check that the value
  output by evaluation is $\hat\phi(Q)$ as claimed.
\end{description}

The two variants only differ in the way the isogeny walk is set up,
and in minor details of the verification; these differences will be
irrelevant to us.

The delay property of this VDF rests, roughly speaking, on the
assumption that a chain of $T$ isogenies of small prime degree $\ell$
cannot be computed more efficiently than by going through each of the
isogenies one at a time, sequentially. %
The case $\ell=2$ is very similar to repeated squaring in groups of
unknown order as used by other VDFs~\cite{Wesolowski,Pietrzak} and
time-lock puzzles~\cite{TLP}: in groups, one iterates $T$ times the
function $x\mapsto x^2$, a polynomial of degree $2$; in isogeny graphs, one
iterates rational fractions of degree $2$. %
See Section~\ref{sec:secure-impl-isog} for more details.

It is important to remark that both setup and evaluation in these VDFs
are ``slow'' algorithms, indeed both need to evaluate an isogeny chain
(either $\phi$, or $\hat\phi$) at one input point of order $N$; this
is in stark contrast with VDFs based on groups of unknown order, where
the setup, and thus its complexity, is independent of the delay
parameter $T$.


\subsection{Instantiation}

The isogeny-based VDF of De Feo \emph{et al.}\ can be understood as a
modification on the Boneh--Lynn--Shacham~\cite{boneh+lynn+shacham04}
signature scheme, where the secret key is replaced by a long chain of
isogenies: signing becomes a ``slow'' operation and thus realizes the
evaluation function, whereas verification stays efficient.

Similarly, we obtain a Delay Encryption scheme by modifying the IBE
scheme of Boneh and Franklin~\cite{doi:10.1137/S0097539701398521}: the
master secret is replaced by a long chain of isogenies, while session
identifiers play the role of identities, so that producing the
decryption key for a given identity becomes a slow operation.

Concretely, setup is identical to that of the VDF. %
A prime of the form $p=4\cdot N\cdot f - 1$ is fixed according to the
security parameter, then setup is actually split into two algorithms:
a $\TSetup$ independent of the delay parameter $T$ and reusable for
arbitrarily many untrusted setups, and a $\Setup$ which depends on
$T$.

\begin{description}
\item[$\TSetup(\lambda)$.]\
  Generate a nearly uniformly random supersingular curve
  $E/\F_p$ by starting from the curve $y^2=x^3+x$ and performing a
  random walk in the $\F_p$-restricted supersingular graph. %
  Output $E$.
\item[$\Setup(E,T)$.]\
  \begin{enumerate}
  \item Perform an $\ell$-isogeny walk $\phi:E\to E'$ of length $T$;
  \item Select a random point $P\in E(\F_p)$ of order $N$, and compute
    $\phi(P)$;
  \item Output $\ek:=(E',\phi)$ and $\pk:=(E',P,\phi(P))$.
  \end{enumerate}
\end{description}

We stress that known homomorphic time-lock puzzles~\cite{C:MalThy19}
also require a one-shot trusted setup. %
Furthermore, unlike constructions based on RSA groups,
there is no evidence that trusted setup is unavoidable for
isogeny-based delay functions, and indeed removing this trusted setup
is an active area of
research~\cite{10.1007/978-3-030-45724-2_18,love2019supersingular}.

The isogeny chain $\phi$ in $\Setup$ can be generated by any of the
two methods proposed by De Feo \emph{et al.}, the difference will be
immaterial for Delay Encryption; as discussed
in~\cite{10.1007/978-3-030-34578-5_10}, a (deterministic) walk limited
to curves and isogenies defined over $\F_p$ will be more efficient,
however a generic (pseudorandom) walk over $\F_{p^2}$ will offer some
partial protection against quantum attacks.

Before defining the other routines, we need two hash functions. %
The first, $H_1:\{0,1\}^*\to E'[N]$, will be used to hash session
identifiers to points of order $N$ in $E'/\F_{p^2}$ (although the
curve $E'$ may be defined over $\F_p$). %
The second, $H_2:\F_{p^2}\to\{0,1\}^\lambda$, will be a key derivation
function. %

\begin{description}
\item[$\Extract(E,E',\phi,\id)$.]\
  \begin{enumerate}
  \item Let $Q = H_1(\id)$;
  \item Output $\hat\phi(Q)$.
  \end{enumerate}
\item[$\Encaps(E,E',P,\phi(P),\id)$.]\
  \begin{enumerate}
  \item Select a uniformly random $r\in\Z/N\Z$;
  \item Let $Q = H_1(\id)$;
  \item Let $k=e_N'(\phi(P),Q)^r$;
  \item Output $(rP,H_2(k))$.
  \end{enumerate}
\item[$\Decaps(E,E',\hat\phi(Q),rP)$.]\
  \begin{enumerate}
  \item Let $k = e_N(rP,\hat\phi(Q))$.
  \item Output $H_2(k)$.
  \end{enumerate}
\end{description}

Correctness of the scheme follows immediately from
Eq.~\eqref{eq:adjoin} and the bilinearity of the pairing. %

\begin{remark}
  Notice that two hashed identities $Q,Q'$ such that
  $Q-Q'\in \langle P\rangle$ are equivalent for encapsulation and
  decapsulation purposes, and thus an adversary only needs to compute
  the image of one of them under $\hat\phi$. %
  However, if we model $H_1$ as a random oracle, the probability of
  two identities colliding remains negligible (about $1/N$).
  
  Alternatively, if $E'$ is defined over $\F_p$, one can restrict the
  image of $H_1$ to $E'[(N,\pi+1)]$, like
  in~\cite{10.1007/978-3-030-34578-5_10}.
\end{remark}


\subsection{Security}

To prove security of their VDF schemes, De~Feo \emph{et al.} defined
the following \emph{isogeny shortcut game}:

\begin{description}
\item[Precomputation.] The adversary receives $N,p,E,E',\phi$, and
  outputs an algorithm $\mathcal{S}$ (in time $\poly(\lambda,T)$).
\item[Challenge.] The challenger outputs a uniformly random
  $Q\in E'[N]$.
\item[Guess.] The algorithm $\mathcal{S}$ is run on input $Q$. The
  adversary wins if $\mathcal{S}$ terminates in time less than
  $\Delta$, and $\mathcal{S}(Q) = \hat\phi(Q)$.
\end{description}

However, it is clear that the $\Delta$-hardness of this game is
insufficient to prove $\Delta$-IND-CPA security of our Delay
Encryption scheme. %
Indeed, while the hardness of the isogeny shortcut obviously
guarantees that the output of $\Extract$ cannot be computed in time
less than $\Delta$, there is at least one other way to decapsulate a
ciphertext $rP$, which consists in evaluating $\phi(rP)$ and computing
$k=e_N'(\phi(rP), Q)$. %
Computing $\phi(rP)$ is expected to be at least as ``slow'' as
computing $\hat\phi(Q)$, however this fact is not captured by the
isogeny shortcut game.

Instead, we define a new security assumption, analogous to the
Bilinear Diffie-Hellman assumption used in standard pairing-based
protocols. %
The \emph{bilinear isogeny shortcut game} is defined as follows:

\begin{description}
\item[Precomputation.] The adversary receives $p,N,E,E',\phi$, and
  outputs an algorithm $\mathcal{S}$.
\item[Challenge.] The challenger outputs uniformly random
  $R\in E[(N,\pi-1)]$ and $Q\in E'[N]$.
\item[Guess.] Algorithm $\mathcal{S}$ is run on $(R,Q)$. The adversary
  wins if $\mathcal{S}$ outputs
  $\mathcal{S}(R,Q) = e_N'(\phi(R),Q) = e_N(R,\hat\phi(Q))$.
\end{description}

We say that the bilinear isogeny shortcut game is $\Delta$-hard if no
adversary running the precomputation in time $\poly(\lambda,T)$
produces an algorithm $\mathcal{S}$ that wins in time less than
$\Delta$ with non-negligible probability. %
The reduction to $\Delta$-IND-CPA of our Delay Encryption scheme
closely follows the proof of security of Boneh and Franklin's IBE
scheme.

\begin{theorem}
  The Delay Encryption scheme presented above is $\Delta$-IND-CPA
  secure, assuming the $\Delta'$-hardness of the bilinear isogeny
  shortcut game, with $\Delta\in\Delta' - o(\Delta')$, when $H_2$ is
  modeled as a random oracle.

  Concretely, suppose there is an efficient $\Delta$-IND-CPA adversary
  $\mathcal{A}$ that has advantage $\epsilon$, and makes $q$ queries
  to $H_2$ (including the queries made by the sub-algorithm
  $\mathcal{D}$). Then there is an efficient algorithm $\mathcal{B}$
  that wins the bilinear isogeny shortcut game with probability at
  least $2\epsilon/q$ and delay $\Delta' = \Delta + q\cdot\poly(\lambda)$.
\end{theorem}
\begin{proof}
  In the precomputation phase, when $\mathcal{B}$ receives the
  parameters $p,N,E,E',\phi$, it draws a random $P\in E(\F_p)$ of
  order $N$, and evaluates $\phi(P)$. %
  It then passes $p,N,E,E',\phi,P,\phi(P)$ to $\mathcal{A}$ for its
  own precomputation phase. %
  Whenever $\mathcal{A}$ makes a call to $H_2(x)$, algorithm
  $\mathcal{B}$ checks whether input $x$ has already been requested,
  in which case it responds with the same answer previously given,
  otherwise it responds with a uniformly sampled output and records
  the query.

  When $\mathcal{A}$ requests its challenge, $\mathcal{B}$ does the
  same, receiving $R\in E[(N,\pi-1)]$ and $Q\in E'[N]$. %
  It then draws a random string $s\in\{0,1\}^\lambda$, and challenges
  $\mathcal{A}$ with the tuple $(R,s,Q)$.

  During the guessing phase, whenever $\mathcal{A}$ (actually,
  $\mathcal{D}$) makes a call to $H_2$, algorithm $\mathcal{B}$
  (actually, $\mathcal{S})$ responds as before. %
  Finally, when $\mathcal{D}$ outputs its guess, $\mathcal{S}$ simply
  returns a random entry among those that were queried to $H_2$.

  Let $\mathcal{H}$ be the event that $\mathcal{A}$ (or $\mathcal{D}$)
  queries $H_2$ on input $e_N(R,\hat\phi(Q))$. %
  We prove that $\Pr(\mathcal{H}) \ge 2\epsilon$, which immediately
  gives the claim of the theorem. %
  To this end, we first prove that $\Pr(\mathcal{H})$ in the
  simulation is equal to $\Pr(\mathcal{H})$ in the real attack; then
  we prove that $\Pr(\mathcal{H})\ge 2\epsilon$ in the real attack.

  To prove the first claim, it suffices to show that the simulation is
  indistinguishable from the real world for $\mathcal{A}$. %
  Indeed, public parameters are distributed identically to a Delay
  Encryption scheme, and the point $R$ that is part of the challenge
  is necessarily a multiple of $P$, since $E[(N,\pi-1)]$ is cyclic. %
  The proof that the two probabilities are equal, then proceeds as in
  \cite[Lemma~4.3, Claim~1]{doi:10.1137/S0097539701398521}.

  The proof that $\Pr(\mathcal{H})\ge 2\epsilon$ is identical to
  \cite[Lemma~4.3, Claim~2]{doi:10.1137/S0097539701398521}. %
  This proves the part of the statement on the winning probability of
  $\mathcal{B}$.
  
  If Algorithm $\mathcal{D}$ runs in time less than $\Delta$,
  algorithm $\mathcal{S}$ runs in time $\Delta$, plus the time
  necessary for drawing the random string $s$ and for answering
  queries to $H_2$. %
  Depending on the computational model, a lookup in the table for
  $H_2$ can take anywhere from $O(1)$ (e.g., RAM model) to $O(q)$
  (e.g., Turing machine). %
  We err on the safe side, and estimate that $\mathcal{S}$ runs in
  time $\Delta + q\cdot\poly(\lambda)$.
\end{proof}


\subsection{Known Attacks}
We now shift our attention to attacks. %
As discussed in~\cite{10.1007/978-3-030-34578-5_10}, there are three
types of known attacks: \emph{shortcut} attacks, discrete logarithm
attacks, and attacks on the computation.

Parameters for a Delay Encryption scheme must be chosen so that all
known attacks have exponential difficulty in the security parameter
$\lambda$. %
Given that (total) attacks successfully compute decapsulation in
exponential time in $\lambda$, it is evident that the delay parameter
$T$ must grow at most subexponentially in $\lambda$.

\paragraph{Shortcut attacks} aim at computing a shorter path
$\psi:E\to E'$ in the isogeny graph from the knowledge of
$\phi:E\to E'$. %
The name should not be confused with the isogeny shortcut game
described above, as shortcut attacks are only one of the possible ways
to beat the game.

De Feo \emph{et al.}\ show that shortcut attacks are possible when the
endomorphism ring of at least one of $E$ or $E'$ is known. %
Indeed, in this case, the isogeny $\phi$ can be translated to an ideal
class in the endomorphism ring, then smoothing techniques similar
to~\cite{kohel2014quaternion} let us convert the ideal to one of
smaller norm, and finally to an isogeny $\psi:E\to E'$ of smaller
degree.

The only way out of these attacks is to select the starting curve $E$
as a uniformly random supersingular curve over $\F_p$, then no
efficient algorithm is known to compute $\End(E)$, nor $\End(E')$. %
Unfortunately, the only way we currently know to sample nearly
uniformly in the supersingular class over $\F_p$, is,
paraphrasing~\cite{galbraith2018computational}, to choose the
endomorphism ring first and then compute $E$ given $\End(E)$.

Thus, the solution put forth in~\cite{10.1007/978-3-030-34578-5_10} is
to generate the starting curve $E$ via a trusted setup, that first
selects $\End(E)$, and then outputs $E$ and throws away the
information about its endomorphism ring. %
We stress that, given a random supersingular curve $E$, computing
$\End(E)$ is a well known hard problem, upon which almost most of
isogeny-based cryptography is founded. %
We explain in the next section how to mitigate the inconvenience of
having a trusted setup, using a distributed protocol.

As stressed in~\cite{10.1007/978-3-030-34578-5_10}, there is no
evidence that ``hashing'' in the supersingular class, i.e., sampling
nearly uniformly without gaining knowledge of the endomorphism ring,
should be a hard problem. %
But there is no evidence it should be easy either, and several
attempts have failed
already~\cite{10.1007/978-3-030-45724-2_18,love2019supersingular}.

Another possibility hinted at in~\cite{10.1007/978-3-030-34578-5_10}
would be to generate ordinary pairing friendly curves with large
isogeny class, as the shortcut attack is then thwarted by the
difficulty of computing the order of the class group of the
endomorphism ring. %
However this possibly seems an even harder problem than hashing to the
supersingular class.

\paragraph{Discrete logarithm attacks} compute $\hat\phi(Q)$ by
directly solving the pairing equation~\eqref{eq:adjoin}. %
In our case, we can even directly attack the key encapsulation. %
Indeed, knowing $rP$, we obtain $r$ through a discrete logarithm, and
then compute $k=e_N'(\phi(P),Q)^r$.

Thanks to the efficiently computable pairing, the discrete logarithm
can actually be solved in $\F_{p^2}$, which justifies taking $p,N$
large enough to resist finite field discrete logarithm computations. %
Obviously, this also shows that our scheme is easily broken by quantum
computers. %
See~\cite{10.1007/978-3-030-34578-5_10}, however, for a discussion of
how a setup with pseudo-random walks over $\F_{p^2}$ resists quantum
attacks in a world where quantum computers are available, but much
slower than classical ones.

\paragraph{Attacks on the computation} do not seek to deviate from the
description of the protocol, but simply try to speed up $\Extract$
beyond the way officially prescribed by the scheme. %
In this sort of attacks, the adversary may be given more resources
than the legitimate user: for example, it may be allowed a very large
precomputation, or it may dispose of an unbounded amount of
parallelism, or it may have access to an architecture not available to
the user (e.g., a quantum computer).

These attacks are the most challenging to analyze, because standard
com\-plexity-theoretical techniques are of little help here. %
% (see, however~\cite{todo})
On some level, this goal is unachievable: given a sufficiently
abstract computational model, and a sufficiently powerful adversary,
any scheme is broken. %
For example, an adversary may precompute all possible pairs
$(Q,\hat\phi(Q))$ and store them in a $O(1)$-accessible RAM, then
extraction amounts to a table lookup. %
However, such an adversary with exponential precomputation,
exponential storage, and constant time RAM is easily dismissed as
unreasonable. %
More subtle trade-offs between precomputation, storage and efficiency
can be obtained, like for example RNS-based techniques to attack
group-based VDFs~\cite{BernsteinSorenson07}. %
However the real impact
of these theoretical algorithms has yet to be determined.

In practice, a pragmatic approach to address attacks on the
computation is to massively invest in highly specialized hardware
development to evaluate the ``sequential and slow'' function quickly,
and then produce the best designs at scale, so that they are available
to anyone who wants to run the extraction. %
This is the philosophy of the competitions organized by
Ethereum~\cite{ethereum-vdf} and Chia~\cite{chia-vdf}, targeting,
respectively, the RSA based VDF and the class group based VDF.

We explore this topic more in detail in
Section~\ref{sec:secure-impl-isog}.


\section{Distributed trusted setup}
\label{sec:distr-trust-setup}

Trusted setup is an obvious annoyance to distributed protocols. %
A way to mitigate this negative impact is to distribute trust over
several participants, ensuring through a multi-party computation that,
if at least one participant is honest, then the setup can be trusted.

Ethereum is notoriously investing in the RSA-based VDF with
Wesolowski's proof~\cite{ethereum-vdf,Wesolowski}, which is known to
require a trusted setup. %
To generate parameters, the Ethereum network will need to run a
distributed RSA modulus generation, for which all available techniques
essentially trace back to the work of Boneh and
Franklin~\cite{10.1007/BFb0052253}.

Distributed RSA modulus generation is notoriously a difficult task:
the cost is relatively high, scales badly with the number of
participants, and the attempts at optimizing it have repeatedly led to
subtle and powerful attacks~\cite{eth-octopus,eth-dogbyte}. %
Worse still, specialized hardware for the delay function must be
designed specifically for the generated modulus, which means that
little design can be done prior to the distributed generation, and
that if the distributed generation is then found to be rigged, a new
round of distributed-generation-then-design is needed.

On the contrary, distributed parameter generation for our Delay
Encryption candidate, or for the isogeny based VDF, is extremely
easy. %
The participants start from a well known supersingular curve with
known endomorphism ring, e.g., $E_0\,:\,y^2=x^3-x$, and repeat, each
at its own turn, the following steps:
\begin{enumerate}
\item Participant $i$ checks all zero-knowledge proofs published by
  participants that preceded them;
\item They perform a pseudorandom isogeny walk $\psi_i:E_{i-1}\to E_i$
  of length $c\log(p)$ in the $\F_p$-restricted supersingular graph;
\item They publish $E_i$, and a zero-knowledge proof that they know an
  isogeny $\psi:E_{i-1}\to E_i$.
\end{enumerate}

The constant $c$ is to be determined as a function of the expansion
properties of the isogeny graph, and is meant to be large enough to
ensure nearly uniform mixing of the walk. %
In practice, this constant is usually small, say $c<10$, implying that
each participant needs to evaluate a few thousands isogenies, a
computation that is expected to take in the order of
seconds~\cite{10.1007/978-3-030-03332-3_15}.

The setup is clearly secure as long as at least one participant is
honest. %
Indeed it is well known that computing a path from $E_i$ to $E_0$ is
equivalent to computing the endomorphism ring of
$E_i$~\cite{kohel2014quaternion,10.1007/978-3-319-78372-7_11}, and,
since $E_i$ is nearly uniformly distributed in the supersingular graph, the
dishonest participants have no advantage in solving this problem
compared to a generic attacker.

This distributed computation scales linearly with the number of
participants, each participant needing to check the proofs of the
previous ones. %
It can be left running for a long period of time, allowing many
participants to contribute trust without any need for prior
registration. %
More importantly, it is \emph{updatable}, meaning that after the
distributed generation is complete, the final curve $E$ can be used as
the starting point for a new distributed trusted setup. %
This way the trusted setup can be updated regularly, building upon the
trust accumulated in previous distributed generations.

Compared with the trusted setup for RSA, the outcome of the setup is
much less critical for the design of hardware. %
Indeed, the primes $p,N$ can be publicly chosen in advance, and
hardware can be designed for them before the trusted setup is
performed. %
The trusted curve $E$ only impacts the first few steps of the ``slow''
isogeny walk $\phi:E\to E'$ generated by the untrusted setup, and can
easily be integrated in the hardware design at a later stage.

\subsection{Proofs of isogeny knowledge}

We take a closer look at the last step each participant takes in the
trusted setup: the proof of isogeny knowledge. %
Ignoring zero-knowledge temporarily, Eq.~\eqref{eq:adjoin} already
provides a proof of knowledge of a non-trivial relation between
$E_{i-1}$ and $E_i$. %
Let $H_1$ be a hash function mapping into
$E_{i-1}[(N,\pi-1)] \times E_i[(N,\pi+1)]$.
Also let $e_N^i$ denote the Weil pairing on $E_i$.
The proof proceeds as follows
\begin{enumerate}
\item Hash the curves $E_{i-1},E_i$ to a pair of points
  $(P,Q) = H_1(E_{i-1},E_i)$;
\item Publish $\psi_i(P), \hat\psi_i(Q)$.
\end{enumerate}
Then verification simply consists of:
\begin{enumerate}
\item Compute $P,Q\gets H_1(E_{i-1},E_i)$,
\item Check that $\psi_i(P)\in E_i[(N,\pi-1)]$ and
  $\hat\psi_i(Q)\in E_{i-1}[(N,\pi+1)]$;
\item Check that $e_N^i(\psi_i(P),Q) = e_N^{i-1}(P,\hat\psi_i(Q))$.
\end{enumerate}

This proof is compact, requiring only four elements of $\F_p$, and
efficient because computing $\psi_i(P),\hat\psi_i(Q)$ only adds a
small overhead to the computation of $\psi_i$, and verification takes
essentially two pairing computations. %

\begin{remark}
  While we believe that an adversary not knowing an isogeny from
  $E_{i-1}$ to $E_i$ has a negligible probability of convincing a
  verifier in the protocol above, it is not clear what kind of knowledge
  is \emph{exactly} proved by it. %
  Ideally, we would like to prove that, given an algorithm that passes
  verification with non negligible probability, one can extract a
  description of some isogeny $\psi':E_{i-1}\to E_{i}$. %

  However, no such algorithm is currently known. %
  Related problems have been studied in the context of cryptanalyses
  of SIDH, under the name of ``torsion point
  attacks''~\cite{10.1007/978-3-319-70697-9_12,cryptoeprint:2019:1291,cryptoeprint:2020:633},
  however these algorithms crucially rely on the knowledge of the
  endomorphism ring of $E_{i-1}$, something we cannot exploit here.

  The only way out is apparently to define a non-falsifiable
  ``knowledge of isogeny'' assumption, which would tautologically
  state that the protocol above is indeed a proof of knowledge of an
  isogeny. %
  We defer investigation of this type of assumptions to future work.
\end{remark}

To turn the protocol above into a zero-knowledge proof, we use an
additional hash function $H_2$ into
$E_i[(N,\pi-1)]\times E_{i-1}[(N,\pi+1)]$, and let
$P',Q' \gets H_2(E_{i-1},E_i)$. %
We now choose $x,y$ secret and publish a NIZK proof for $(x,y)$ satisfying
$$ e_N^i(X,Q) e_N^{i-1}(P,Q')^y  = e_N^{i-1}(P,Y) e_N^i(P',Q)^x . $$
More precisely, we publish:
\begin{itemize}
\item two Pedersen commitments 
 $X = x P' + \psi_i(P) \in E_i[(N,\pi-1)]$ and
 $Y = y Q' + \hat\psi_i(Q) \in E_{i-1}[(N,\pi+1)]$,
\item two public keys $Y' = e_N^{i-1}(P,Q')^y$ and $X' = e_N^i(P',Q)^x$, and
\item two Schnorr proofs of knowledge $(c,s_X,s_Y)$ for $x$ of $X'$ and $y$ of $Y'$
over the base points $e_N^{i-1}(P,Q')$ and $e_N^i(P',Q')$, respectively.
%TODO: Referee says:
% - p.15, last line: write "what kind of knowledge".
% but we say knowledge for x and y already?
\end{itemize}
At this point, our verifier now checks 
\begin{itemize}
\item $e_N^i(X,Q) Y' = e_N^{i-1}(P,Y) X'$,
\item non-triviality $X' \ne e_N^i(X,Q)$ and $Y' \ne e_N^{i-1}(P,Y)$ of the commitments, and
\item the proofs of knowledge $s_X,s_Y$ using
$$ c = H( 
  e_N^{i-1}(P,Q') \| e_N^i(P',Q) \| 
  X' \| Y' \| 
  (X')^c e_N^{i-1}(P,Q')^{s_X} \| 
  (Y')^c e_N^i(P',Q)^{s_Y} 
). $$
\end{itemize}
In this, we ask verifiers to compute four pairings, which only doubles
the verifier time.
% Also, four hash-to-curve operations, and 
% four-ish scalar multiplications in the target group.
% ($e_N^i(P',Q)$, $e_N^{i-1}(P,Q')$, $e_N^i(X,Q)$, $e_N^{i-1}(P,Y)$)

We establish with the proof of knowledge that $X'$ and $Y'$ have
the desired form, assuming CDH in the target group $G_T\subset\F_{p^2}$,
and modeling $H,H_1,H_2$ as random oracles.
% https://www.di.ens.fr/david.pointcheval/Documents/Papers/2000_joc.pdf
% see: https://crypto.stackexchange.com/questions/48616/prove-the-security-of-schnorrs-signature-scheme
% I doubt https://eprint.iacr.org/2013/418.pdf helps
We then apply the non-triviality check to deduce that some such 
 $X - x P' = \psi_i(P) \in E_i[(N,\pi-1)]$ and
 $Y - y Q' = \hat\psi_i(Q) \in E_{i-1}[(N,\pi+1)]$
exist, thanks to the strong bilinear Diffie-Hellman assumption (SBDH) 
\cite{10.1007/978-3-540-74143-5_24}. 
% see https://crypto.stackexchange.com/questions/51729/q-strong-bilinear-diffie-hellman
It now follows from bilinearity that $e_N^i(X,Q)Y' = e_N^{i-1}(P,Y)X'$, as desired.

We learn nothing about $X$ and $Y$ except for this pairing equation 
because Pedersen commitments are perfectly blinding and
the Schnorr proof is zero-knowledge in the ROM.

\smallskip

For completeness, we also mention some other tools with which one
might prove knowledge of this isogeny in zero knowledge, although none
seem to be competitive with the technique above.
 
First, there exists a rapidly expanding SNARK toolbox from which
one could perform $\F_p$ arithmetic inside the SNARK to check the
verification of the second and third conditions directly.  
As instantiating the delay function imposes restrictions on $p$,
one cannot necessarily select $p$ using the Cocks-Pinch method to
provide a pairing friendly elliptic curve with group order $p$, 
like in \cite{ZEXE}. % https://eprint.iacr.org/2018/962.pdf
There are optimisations for arithmetic in arbitrary $\F_p$ 
however, especially using polynomial commitments,
like in \cite{plookup}. % https://eprint.iacr.org/2020/315.pdf

Second, there are well known post-quantum isogeny-based proofs:
\begin{description}
\item[SIDH-style proofs~\cite{defeo+jao+plut12}] %
  are very inconvenient, because they require primes of a specific
  form, and severely limit the length of pseudo-random walks. %
  On top of that, they are very inefficient, and do not have perfect
  zero-knowledge. %
\item[SeaSign-style proofs~\cite{10.1007/978-3-030-17659-4_26}] %
  have sizes in the hundreds of kilobytes, and their generation and
  verification are extremely slow (dozens of hours). %
  Note that several of the optimizations used for signatures,
  including the class group order precomputation of
  CSI-FiSh~\cite{10.1007/978-3-030-34578-5_9}, are not available in
  this context. %
  More research on the optimization of SeaSign-style proofs for this
  specific context would be welcome.
\end{description}


\section{Watermarking}
\label{sec:watermarking}

A common requirement in cryptocurrencies is to be able to reward
participants who spend resources to compute the delay function, be it
in the context of a VDF or a Delay Encryption. %
Wesolowski~\cite{Wesolowski} introduced the concept of \emph{proof
  watermarking}, i.e., attaching the proof of a VDF evaluation to an
identity, so that the ownership of the proof cannot be usurped without
performing essentially the same work as evaluating the VDF normally.

In the context of isogeny based VDFs, or of extraction in Delay
Encryption, this is a meaningless concept, because there is simply no
proof to watermark. %
Nevertheless, it is possible to attach a watermark to the output of
the delay function, which gives evidence that the owner of the
watermark spent an amount of effort comparable to legitimately
computing the output. %
The idea is to publish a \emph{mid-point} update on the progress of
the evaluation, and attach this mid-point to the identity of the
evaluator.

Concretely, given parameters $\phi:E\to E'$ and $(P,\phi(P))$, the
isogeny walk is split into two halves of equal size
$\phi_1:E\to \Emid$ and $\phi_2:\Emid\to E'$ so that
$\phi=\phi_2\circ\phi_1$, and $\phi_1(P)$ is added to the public parameters. %
Each evaluator then generates a secret key $s\in\Z/N\Z$ and a public
key $S = s \phi(P)$. %
When evaluating $\hat\phi=\hat\phi_1\circ\hat\phi_2$ at a point
$Q\in E'[N]$, the evaluator:
\begin{enumerate}
\item Computes $\Qmid=\hat\phi_2(Q)$,
\item Computes and publishes $s\Qmid$,
\item Finishes off the computation by computing
  $\hat\phi(Q)=\hat\phi_1(\Qmid)$.
\end{enumerate}
A watermark can then be verified by checking that
\[e_N^\mathrm{mid}(\phi_1(P),s\Qmid) = e_N'(S,Q).\]
Interestingly, this proof is \emph{blind}, meaning that it can be
verified even before the work is finished.

Given $\hat\phi(Q)$, a usurper wanting to claim the computation for
themselves would need to either start from $Q$ and compute
$\hat\phi_2(Q)$, or start from $\hat\phi(Q)$ and compute
$\frac{\phi_1(\hat\phi(Q))}{\deg\phi_1}$. %
Either way, they would perform at least half as much work as if they
had legitimately evaluated the function.

It is possible, nevertheless, for a usurper to
target a specific evaluator, by generating a random $u\in\Z/N\Z$, and
choosing $us\phi_1(P)$ as public key. %
Then, any proof $s\Qmid$ for the legitimate evaluator is easily
transformed to a proof $us\Qmid$ for the usurper. %
This attack is easily countered by having all evaluators publish a
zero-knowledge proof of knowledge of their secret exponent $s$, along
with their public key $s\phi_1(P)$. %


\section{Challenges in implementing isogeny-based delay functions}
\label{sec:secure-impl-isog}

For a delay function to be useful, there need to be convincing
arguments as to why the evaluation cannot be performed considerably
better than with the legitimate algorithm.

In this sense, repeated squaring modulo an RSA modulus is especially
appealing: modular arithmetic has been studied for a long time, and we
are reasonably confident that we know all useful algorithms and
hardware in this respect; and the repeated application of the function
$x\mapsto x^2$ is so simple that one may hope no better algorithm
exists (see~\cite{BernsteinSorenson07}, though).

Repeated squaring in class groups, already, raises more skepticism, as
the arithmetic of class groups is a much less studied area. %
This clearly had an impact on Ethereum's choice to go with RSA-based
VDFs, despite class group based ones not needing a trusted setup.

For isogeny based delay functions, we argue that the degree of
assurance seems to be nearly as good as for RSA based ones, although
more research is certainly needed. %
To support this claim, we give here more details on the way the
evaluation of $\hat\psi$ is performed, that were omitted
by~\cite{10.1007/978-3-030-34578-5_10}.

For a start, we must choose a prime degree $\ell$. %
Intuitively, the smaller, the better, thus we shall fix $\ell=2$,
although $\ell=3$ also deserves to be studied. %
A $2$-isogeny is represented by rational maps of degree $2$, thus we
expect one isogeny evaluation to require at least one multiplication
modulo $p$. %
Our goal is to get as close as possible to this lower bound, by
choosing the best representation for the elliptic curves, their
points, and their isogenies.

It is customary in isogeny based cryptography to use curves in
Montgomery form, and projective points in $(X:Z)$ coordinates, as
these give the best formulas for arithmetic operations and
isogenies~\cite{costello2016sidh,10.1007/978-3-319-79063-3_11}. %
Montgomery curves satisfy the equation
\[E \;:\; y^2 = x^3 + Ax^2 + x,\] %
in particular they have a point of order two in $(0,0)$, and two other
points of order two with $x$-coordinates $\alpha$ and $1/\alpha$,
where $\alpha$ is a root of the polynomial $x^2+Ax+1$, and possibly
lives in $\F_{p^2}$. %
These three points define the three possible isogenies of degree $2$
starting from $E$. %
The Montgomery form is almost unique, there being only six possible
choices for the $A$ coefficient for a given isomorphism class,
corresponding to the three possible choices for the point to send in
$(0,0)$ (each taken twice).

In our case, all three points (in projective coordinates) $(0:1)$,
$(\alpha:1)$ and $(1:\alpha)$, are defined over $\F_p$, we thus choose
to distinguish one additional point by writing the curves as
\[E_\alpha \;:\; y^2 = x(x-\alpha)(x - 1/\alpha),\] %
with $\alpha\ne0,\pm 1$. %
We call this a \emph{semi-Montgomery form}; although it is technically
equivalent to the Montgomery form, $2$-isogeny formulas are expressed
in it more easily. %
Recovering the Montgomery form is easy via $A=-\alpha-1/\alpha$.

Using the formula of Renes~\cite{10.1007/978-3-319-79063-3_11}, we
readily get the isogeny with kernel generated by
$(\alpha:1)$ as
\begin{equation}
  \label{eq:isog-forward}
  \phi_\alpha(x,y) = \left(x\frac{x\alpha - 1}{x - \alpha}, \dots\right),
\end{equation}
and its image curve is the Montgomery curve defined by
$A = 2-4\alpha^2$. %
By comparing with the multiplication-by-$2$ map on $E_\alpha$, we
obtain the dual map to $\phi$ as
\begin{equation}
  \label{eq:isog-backward}
  \hat\phi_\alpha(x,y) = \left(\frac{(x+1)^2}{4\alpha x}, \dots\right).
\end{equation}
It is clear from this formula that the kernel of $\hat\phi_\alpha$ is
generated by $(0,0)$.

This formula is especially interesting, as we verify that its
projective version in $(X:Z)$ coordinates only requires $2$
multiplications and $1$ squaring:
\begin{equation}
  \label{eq:isog-proj}
  \hat\phi_\alpha(X:Z) = \bigl((X+Z)^2 : 4\alpha XZ\bigr),
\end{equation}
and the squaring can be performed in parallel with one
multiplication. %
The analogous formulas for $\phi_{1/\alpha}$ are readily obtained by
replacing $\alpha\to 1/\alpha$ in the previous ones, and moving around
projective coefficients to minimize work.

But, if we want to chain $2$-isogenies, we need a way to compute the
semi-Montgomery form of the image curve. %
For the given $A=4\alpha^2-2$, direct calculation shows that the two
possible choices are
\begin{equation}
  \label{eq:next-curve}
  \alpha' = 2\alpha\left(\alpha \pm \sqrt{\alpha^2 - 1}\right) - 1
  = \left(\alpha \pm \sqrt{\alpha^2-1})\right)^2.
\end{equation}
As we know that $(0,0)$ generates the dual isogeny to $\phi_\alpha$,
neither choice of $\alpha'$ will define a backtracking walk. %
Interestingly, Castryck and Decru~\cite{cryptoeprint:2019:1404} show
that when $p=7\mod 8$, if $\alpha\in\F_p$, $\phi_\alpha$ is a
horizontal isogeny (see definition in~\cite{cryptoeprint:2019:1404}),
and $\alpha'$ is defined as
\[\alpha' = \left(\alpha + \sqrt{\alpha^2-1})\right)^2\]
where $\sqrt{\alpha^2-1}$ denotes the principal square root, then
$\alpha'\in\F_p$ and $\phi_{\alpha'}$ is horizontal too. %
This gives a very simple algorithm to perform a non-backtracking
$2$-isogeny walk staying in the $\F_p$-restricted isogeny graph, i.e.,
a walk on the \emph{$2$-crater}. %
Alternatively, if a pseudo-random walk in the full supersingular graph
is wanted, one simply takes a random square root of $\alpha^2-1$.

Using these formulas, the isogeny walk $\phi:E\to E'$ is simply
represented by the list of coefficients $\alpha$ encountered, and the
evaluation of $\hat\phi$ using Formula~\eqref{eq:isog-proj} costs $2$
multiplications and $1$ parallel squaring per isogeny.

\subsubsection{Implementation challenges.}
Following the recommendations of~\cite{10.1007/978-3-030-34578-5_10},
for a 128-bits security level we need to choose a prime $p$ of around
$1500$ bits, which is comparable to the 2048-bits RSA arithmetic
targeted by Ethereum, although possibly open to optimizations for
special primes.

In software, the latency of multiplication modulo such a prime is
today around 1$\mu$s. %
The winner of the Ethereum FPGA competition~\cite{ethereum-vdf}
achieved a latency of 25ns for 2048-bits RSA arithmetic. %
Assuming a pessimistic baseline of 50ns for one $2$-isogeny
evaluation, for a target delay of $1$ hour we need an isogeny walk of length
$\approx 7\cdot 10^{10}$. %
That represents as many coefficients $\alpha$ to store, each occupying
$\approx 1500$ bits, i.e., $\approx 16$TiB of storage!

We stress that only evaluators need to store that much information,
however any FPGA design for isogeny-based delay functions must take
this constraint into account, and provide fast storage with
throughputs of the order of several GiB/s. %
% While such throughputs may be achieved by attaching several solid
% state drives in parallel, this solution may turn out to be unpractical
% and expensive.

At present, we do not know any configuration that pushes these
2-isogeny computations into being memory bandwidth bound. 
In fact, computational adversaries only begin encountering current
DRAM and CPU bus limits when going an order of magnitude faster
than the hypothetical high speeds above.

An isogeny-based VDF could dramatically reduce storage requirements by
doing repeated shorter evaluations, and simply hashing each output
to be the input for the next evaluation.  We sacrifice verifier time
by doing so, but verifiers remain fast since they still only compute
two pairings.  We caution however
that this trick does not apply to Delay Encryption. 

In~\cite{10.1007/978-3-030-34578-5_10}, De Feo \emph{et al.} describe
an alternative implementation that divides the required storage by a
factor of $1244$, at the cost of slowing down evaluation by a factor
of at least $\log_2(1244)$. %
Unfortunately this trade-off seems unacceptable for applications where
the evaluator wants to get to the result as quickly as possible. %

It would be very interesting to find compact representations of very
long isogeny chains which do not come at the expense of efficiently
evaluating them.

\subsubsection{Optimality}
Formula~\eqref{eq:isog-proj} is, intuitively, almost optimal, as we
expect that a $2$-isogeny in projective $(X:Z)$ coordinates should
require at least $2$ multiplications. %
And indeed we know of at least one case where a $2$-isogeny can be
evaluated with $2$ parallel multiplications: the isogeny of kernel
$(0:1)$ is given by
\begin{equation}
  \label{eq:isog-special}
  \phi_0(x,y) = \left(\frac{(x - 1)^2}{x}, \dots\right),
\end{equation}
or, in projective coordinates,
\begin{equation}
  \label{eq:isog-special-proj}
  \phi_0(X:Z) = \bigl((X - Z)^2:XZ\bigr),
\end{equation}
which only requires one parallel multiplication and squaring.

We tried to construct elliptic curve models and isogeny formulas that
could evaluate $2$-isogeny chains using only $2$ parallel
multiplications per step, however any formula we could find had a
coefficient similar to $\alpha$ intervene in it, and thus bring the
cost up by at least one multiplication.

Intuitively, this is expected: there are exponentially many isogeny
walks, and the coefficients $\alpha$ must necessarily intervene in the
formulas to distinguish between them. %
However this is far from being a proof. %
Even proving a lower bound of $2$ \emph{parallel} multiplications
seems hard.

It would be interesting to prove that any $2$-isogeny chain needs at
least $2$ \emph{sequential} multiplications for evaluation, or
alternatively find a better way to represent and evaluate isogeny
chains.


\section{Conclusion}

We introduced a new time delay primitive, named Delay Encryption,
related to Time-lock Puzzles and Verifiable Delay Functions. %
Delay Encryption has some interesting applications such as sealed-bid
auctions and electronic voting. %
We gave an instantiation of Delay Encryption using isogenies of
supersingular curves and pairings, and discussed several related
topics that also apply to the VDF of De Feo, Masson, Petit and Sanso.

Several interesting questions are raised by our work, such as, for
example, clarifying the relationship between Delay Encryption,
Verifiable Delay Functions and Time-lock puzzles.

Like the isogeny-based VDF, our Delay Encryption requires a trusted
setup. %
We described an efficient way to perform a distributed trusted setup,
however the associated zero-knowledge property relies on a
non-falsifiable assumption which requires more scrutiny.

The implementation of delay functions from isogenies presents several
practical challenges, such as needing very large storage for the
public parameters. %
On top of that, it is not evident how to prove the optimality of
isogeny formulas used for evaluating the delay function. %
While we gave here extremely efficient formulas, these seem to be at
least one multiplication more expensive than the theoretical
optimum. %
More research on the arithmetic of elliptic curves best adapted to
work with extremely long chains of isogenies is needed.

Finally, we invite the community to look for more constructions of
Delay Encryption, in particular quantum-resistant ones.

\def\doi#1{\href{https://doi.org/#1}{\tt https://doi.org/\nolinkurl{#1}}}
\bibliography{isovdf,zkp,isogenies_bib/isogenies}
\bibliographystyle{splncs04}

\end{document}

% LocalWords:  bilinear instantiation VDF subexponential morphisms
% LocalWords:  instantiations supersingular endomorphism morphism
% LocalWords:  isogenous homomomorphism endomorphisms homomorphism
% LocalWords:  isogenies Frobenius isogeny subgraphs distorsion
% LocalWords:  prover soundess sequentiality quaternion projective

% LocalWords:  homomorphic
