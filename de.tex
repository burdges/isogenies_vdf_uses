\documentclass{llncs}

\usepackage{amsmath,amsfonts,amssymb}
\usepackage{xcolor}
\usepackage{hyperref}
\hypersetup{
	unicode=true,
	colorlinks=true,
	citecolor=blue!70!black,
	filecolor=black,
	linkcolor=red!70!black,
	urlcolor=blue,
	pdfstartview={FitH},
}

\newcommand{\Z}{\mathbb{Z}}
\newcommand{\F}{\mathbb{F}}
\renewcommand{\O}{\mathcal{O}}
\DeclareMathOperator{\poly}{poly}
\DeclareMathOperator{\polylog}{polylog}
\DeclareMathOperator{\Setup}{\mathsf{Setup}}
\DeclareMathOperator{\TSetup}{\mathsf{TrustedSetup}}
\DeclareMathOperator{\Extract}{\mathsf{Extract}}
\DeclareMathOperator{\Encaps}{\mathsf{Encaps}}
\DeclareMathOperator{\Decaps}{\mathsf{Decaps}}
\newcommand{\pp}{\mathsf{pp}}
\newcommand{\id}{\mathsf{id}}
\newcommand{\idk}{\mathsf{idk}}
\newcommand{\keyspace}{\mathcal{K}}
\newcommand{\cipherspace}{\mathcal{C}}

\title{Delay Encryption}
\author{Jeffrey Burdges\inst{1}
  \and
  Luca De Feo\inst{2}\orcidID{0000-0002-9321-0773}}
\institute{
  Web 3, Switzerland
  \and
  IBM Research Zürich, Switzerland
}

\begin{document}

\maketitle

\begin{abstract}
  We introduce a new primitive named Delay Encryption, and give an
  efficient instantation based on isogenies of supersingular curves
  and pairings. %
  Delay Encryption is related to Time-lock Puzzles and Verifiable
  Delay Functions, and can be roughly described as ``identity based
  encryption with slow decryption''. %
  It has several applications in distributed protocols, such as
  sealed bid Vickrey auctions and...

  We give an instantiation of Delay Encryption by modifying Boneh and
  Frankiln's IBE scheme, where we replace one secret by a long chain
  of isognies, as in the isogeny VDF of De Feo, Masson, Petit and
  Sanso. %
  Similarly to the isogeny-based VDF, our Delay Encryption requires a
  trusted setup before parameters can be safely used; our trusted
  setup is identical to that of the VDF, thus the same parameters can
  be generated once and shared for many executions of both protocols,
  with possibly different delay parameters.

  We also discuss several topics around delay protocols
  based on isogenies that were left untreated by De Feo \emph{et al.},
  namely: distributed trusted setup, watermarking, and implementations
  issues.
\end{abstract}

\section{Introduction}
\label{sec:introduction}

% Recall that an Identity Based Encryption scheme (IBE) is a public
% key encryption

\subsection{Applications of Delay Encryption}


\section{Definitions}
\label{sec:definitions}

Our definition of Delay Encryption uses an API similar to a Key
Encapsulation Mechanism; the adaptation to a PKE-like API is
straightforward. A Delay Encryption scheme consists of four
algorithms: $\Setup$, $\Extract$, $\Encaps$ and $\Decaps$:

\begin{description}
\item[$\Setup(\lambda, T) \to \pp$.] %
  Takes a \emph{security parameter} $\lambda$, a \emph{delay
    parameter} $T$, and produces a set of public parameters $\pp$. %
  $\Setup$ must run in time $\poly(\lambda,T)$.
\item[$\Extract(\pp,\id) \to \idk$.] %
  Takes the public parameters $\pp$ and a \emph{session identifier}
  $\id\in\{0,1\}^*$, and outputs a \emph{session key} $\idk$. %
  $\Extract$ is expected to run in time \emph{exactly} $T$, see below.
\item[$\Encaps(\pp,\id)\to (c,k)$.] %
  Takes the public parameters $\pp$ and a \emph{session identifier}
  $\id\in\{0,1\}^*$, and outputs a \emph{cyphertext}
  $c\in\cipherspace$ and a \emph{key} $k\in\keyspace$. %
  $\Encaps$ must run in time $\poly(\lambda)$.
\item[$\Decaps(\pp,\id,\idk,c)\to k$.] %
  Takes the public parameters $\pp$, a \emph{session identifier}
  $\id$, a \emph{session key} $\idk$, a ciphertext $c\in\cipherspace$,
  and outputs key $k\in\keyspace$. %
  $\Decaps$ must run in time $\poly(\lambda)$.
\end{description}

A Delay Encryption scheme is correct if
\[\bigl(c,\Decaps(\pp,\id,\idk,c)\bigr) = \Encaps(\pp,\id),\]
whenever $\idk=\Extract(\pp,\id)$. %
The security of Delay Encryption is defined similarly to that of
public key encryption schemes, and in particular of identity-based
ones; however one additional property is required of $\Extract$: that
for a randomly selected identifier $\id$, the probability that
$\Extract$ outputs $\idk$ in time less than $T$ is negligible. %
We now give the formal definition.

\paragraph{The security game.} It is apparent from the definitions
that Delay Encryption has no secrets: after public parameters $\pp$
are generated, anyone can run any of the algorithms. %
Thus, the usual notion of indistinguishability will only be defined
with respect to the delay parameter $T$: no adversary is able to
distinguish a key $k$ from a random string in time $T-o(t)$, but
anyone can in time $T$. %
Properly defining what is meant by ``time'' requires fixing a
computation model. %
Here we follow the usual convention from VDFs, and assume a model of
parallel computation: in this context, ``time $T$'' may mean $T$ steps
of a parallel Turing machine, or an arithmetic circuit of depth $T$. %
Crucially, we do not bound the amount of parallelism of the Turing
machine, or the breadth of the circuit, i.e., we focus on
\emph{sequential delay} functions.

We consider the following $\Delta$-IND-CCA game. %
Note that the game involves no oracles, owing to the fact that the
scheme has no secrets.
%
\begin{description}
\item[Precomputation.] The adversary receives $\pp$ as input, and
  outputs an algorithm $\mathcal{D}_1$. %
\item[Challenge.] The challenger selects a random $\id$ and computes
  $(c,k_0)\gets\Encaps(\pp,\id)$. %
  It then picks a uniformly random $k_1\in\keyspace$, and a random bit
  $b\in\{0,1\}$. %
  Finally, it outputs $(c,k_b)$.
\item[Guess.] The algorithm $\mathcal{D}_1$ is run on input
  $(c,k_b)$. %
  The adversary wins if $\mathcal{D}_1$ terminates in time less than
  $\Delta$, and the output is such that $\mathcal{D}_1(c,k_b) = b$.
\end{description}

We say a Delay Encryption scheme is \emph{$\Delta$-Delay
  Indistinguishable under Chosen Ciphertext Attacks} if, for any
efficient adversary running the precomputation in time
$\poly(\lambda,T)$, the probability of winning the game is
negligible. %
Obviously, the interesting schemes are those where $\Delta = T-o(T)$.


\subsection{Isogenies as delay functions}

We instantiate Delay Encryption from the same framework De Feo,
Masson, Petit and Sanso used to instantiate Verifiable Delay
Functions~\cite{10.1007/978-3-030-34578-5_10}. %
We briefly recall it here for completeness.

An elliptic curve $E$ over a finite field $\F_{p^n}$ is said to be
supersingular if the trace of its Frobenius endomorphism is divisible
by $p$, i.e., if $\#E(\F_{p^n})=1\mod p$. %
Over the algebraic closure of $\F_p$, there is only a finite number of
isomorphism classes of supersingular curves, and every class contains
a curve defined over $\F_{p^2}$.

An isogeny is a group morphism of elliptic curves with finite
kernel. %
In particular, isogenies preserve the group order of elliptic curves,
and thus they preserve supersingularity. %
Isogenies can be represented by ratios of polynomials, and, like
polynomials, have a \emph{degree}. %
Isogenies of degree $\ell$ are also called $\ell$-isogenies; the
degree is multiplicative with respect to composition, thus
$\deg\phi\circ\psi=\deg\phi\cdot\deg\psi$. %
The degree is an important invariant of isogenies, roughly measuring
the amount of information needed to represent them.

An isogeny graph is a graph whose vertices are isomorphism classes of
elliptic curves, and whose edges are isogenies, under some
restrictions. %
Isogeny-based cryptography mainly uses two types of isogeny graphs:
\begin{itemize}
\item The \emph{full supersingular graph} of $\F_p$, whose vertices
  are all isomorphism classes of supersingular curves over $\F_{p^2}$,
  and whose edges are all isogenies of a prime degree $\ell$;
  typically $\ell=2,3$.
\item The \emph{$\F_p$-restricted supersingular graph}, or
  \emph{supersingular CM graph} of $\F_p$, whose vertices are all
  $\F_p$-isomorphism classes of supersingular curves over $\F_p$, and
  whose edges are $\ell$-isogenies for all primes $\ell$ up to some
  bound; typically $\ell\lessapprox\lambda\log\lambda$, where
  $\lambda$ is the security parameter.
\end{itemize}

Any $\ell$-isogeny $\phi:E\to E'$ has a unique \emph{dual}
$\ell$-isogeny $\hat\phi:E'\to E$ such that
\begin{equation}
  \label{eq:adjoin}
  e_N(\phi(P),Q) = e_N'(P,\hat\phi(Q)),
\end{equation}
for any integer $N$ and any points $P\in E[N]$, $Q\in E'[N]$, where
$e$ is the Weil pairing on $E$, and $e'$ the one on $E'$. %
The same equation, with the same $\hat\phi$, also holds for any other
known pairing, such as the Tate and Ate pairings.

The framework of De Feo \emph{et al.} uses chains of small degree
isogenies as delay functions, and the pairing
equation~\eqref{eq:adjoin} as an efficient means to verify the
computation. %
Formally, they propose two related instantiations of VDF, following
the same pattern:
%
\begin{description}
\item[Trusted setup] takes as input a security parameter $\lambda$,
  and produces a prime $N\approx 2^{2\lambda}$, a prime $p=N\cdot f+1$
  for some cofactor $f\sim 2^{\lambda^3}$, and a random supersingular
  elliptic curve $E$ over $\F_p$.
\item[Setup] takes as input $p,N,E$, a delay parameter $T$, and
  performs a walk in an $\ell$-isogeny graph to produce a degree
  $\ell^T$ isogeny $\phi:E\to E'$.
  
  It also computes a point $P\in E$ of order $N$, and outputs
  $P,\phi(P)$.
\item[Evaluation] takes as input a random point $Q\in E'[N]$ and outputs
  $\hat\phi(Q)$.
\item[Verification] uses Eq.~\eqref{eq:adjoin} to check that the value
  output by evaluation is $\hat\phi(Q)$ as claimed.
\end{description}

The two variants only differ in the way the isogeny walk is set up,
and in minor details of the verification; these differences will be
irrelevant to us.

The delay property of this VDF rest, roughly speaking, on the
assumption that a chain of $T$ isogenies of small prime degree $\ell$
cannot be computed more efficiently than by going through each of the
isogenies one at a time, sequentially. %
The case $\ell=2$ is very similar to repeated squaring in groups of
unknown order as used by other VDFs~\cite{Wesolowski,Pietrzak} and
time-lock puzzles~\cite{TLP}: in the latter one iterates $T$ times the
function $x\mapsto x^2$, a polynomial of degree $2$; in the former one
iterates rational fractions of degree $2$. %
See Section~\ref{sec:secure-impl-isog} for more details.

It is important to remark that both setup and evaluation in these VDFs
are ``slow'' algorithms, indeed both need to evaluate an isogeny chain
(either $\phi$, or $\hat\phi$) at one input point of order $N$; this
is in stark contrast with VDFs based on groups of unknown order, where
the complexity of setup does not depend on the delay parameter $T$.


\subsection{Instantiation of Delay Encryption}

The isogeny-based VDF of De Feo \emph{et al.}\ can be understood as a
modification on the Boneh--Lynn--Shacham~\cite{boneh+lynn+shacham04}
signature scheme, where the secret key is replaced by a long chain of
isogenies: signing becomes a ``slow'' operation and thus realizes the
evaluation function, whereas verification stays efficient.

Similarly, we obtain a Delay Encryption scheme by modifying the IBE
scheme of Boneh and Franklin~\cite{10.1007/3-540-44647-8_13}: the
master secret is replaced by a long chain of isogenies, while session
identifiers play the role of identities, so that producing the
decryption key for a given identity becomes a slow operation.

Concretely, $\Setup$ is identical to that of the VDF: it is split in a
trusted and an untrusted part, the former being independent of the
delay parameter $T$, and reusable for arbitrarily many untrusted
setups.

\begin{description}
\item[$\TSetup(\lambda)$.]\
  \begin{enumerate}
  \item Generate primes $N$ and $p=4\cdot N\cdot f - 1$, such that
    discrete logarithms in the subgroup of order $N$ of $\F_{p^2}$ are
    $\lambda$-hard;
  \item Generate a nearly uniformly random supersingular curve
    $E/\F_p$ by starting from the curve $y^2=x^3+x$ and performing a
    random walk in the $\F_p$-restricted supersingular graph;
  \item Output $N,p,E$.
  \end{enumerate}
\item[$\Setup(N,p,E,T)$.]\
  \begin{enumerate}
  \item Perform an $\ell$-isogeny walk $\phi:E\to E'$ of length $T$;
  \item Select a random point $P\in E(\F_p)$ of order $N$, and compute
    $\phi(P)$;
  \item Output $E',\phi,P,\phi(P)$.
  \end{enumerate}
\end{description}

The isogeny chain $\phi$ in $\Setup$ can be generated by any of the
two methods proposed by De Feo \emph{et al.}, the difference will be
immaterial for Delay Encryption; as discussed
in~\cite{10.1007/978-3-030-34578-5_10}, a (deterministic) walk limited
to curves and isogenies defined over $\F_p$ will be more efficient,
however a generic (pseudorandom) walk over $\F_{p^2}$ will offer some
partial protection against quantum attacks.

Before defining the other routines, we need two hash functions. %
The first, $H_1:\{0,1\}*\to E'[N]$, will be used to hash session
identifiers to points of order $N$ in $E'/\F_{p^2}$ (although the
curve $E'$ may be defined over $\F_p$). %
The second, $H_2:F_{p^2}\to\{0,1\}^\lambda$, will be a key derivation
function. %
Let $\pp = (N,p,E,E')$, then we define:

\begin{description}
\item[$\Extract(\pp,\phi,\id)$.]\
  \begin{enumerate}
  \item Let $Q = H_1(\id)$;
  \item Output $\hat\phi(Q)$.
  \end{enumerate}
\item[$\Encaps(\pp,P,\phi(P),\id)$.]\
  \begin{enumerate}
  \item Select a uniformly random $r\in\Z/N\Z$;
  \item Let $Q = H_1(\id)$;
  \item Let $k=e_N'(\phi(P),Q)^r$;
  \item Output $(rP,H_2(k))$.
  \end{enumerate}
\item[$\Decaps(\pp,\hat\phi(Q),rP)$.]\
  \begin{enumerate}
  \item Let $k = e_N(rP,\hat\phi(Q))$.
  \item Output $H_2(k)$.
  \end{enumerate}
\end{description}

Correctness of the scheme follows immediately from
Eq.~\eqref{eq:adjoin} and the bilinearity of the pairing. %
We next address its security.

\subsection{Security}



\begin{remark}
  Notice that two (hashed) identities $Q,Q'$ such that
  $Q-Q'\in\langle P\rangle$ are equivalent for encapsulation and
  decapsulation purposes; however, thanks to $H_1$, the probability of
  two identities collisioning this way is negligible (about $1/N$).
  
  Alternatively, if $E'$ is defined over $\F_p$, one can restrict the
  image of $H_1$ to the subgroup of $E'[N]$ annihilated by $\pi+1$,
  where $\pi$ is the Frobenius endomorphsim, like
  in~\cite{10.1007/978-3-030-34578-5_10}.
\end{remark}

\section{Distributed trusted setup}
\label{sec:distr-trust-setup}

\section{Watermarking}
\label{sec:watermarking}

\section{Challenges in implementing isogeny-based delay functions}
\label{sec:secure-impl-isog}

\section{Conclusion}

\bibliography{isovdf,isogenies_bib/isogenies}
\bibliographystyle{splncs04}

\end{document}

% LocalWords:  bilinear instantiation VDF subexponential morphisms
% LocalWords:  instantiations supersingular endomorphism morphism
% LocalWords:  isogenous homomomorphism endomorphisms homomorphism
% LocalWords:  isogenies Frobenius isogeny subgraphs distorsion
% LocalWords:  prover soundess sequentiality quaternion

